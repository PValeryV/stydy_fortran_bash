\title{ВВедение}

Одна и таже инофрмация из входного файла вводится по разному в оперативную память с целью освоения работы с различными структурами данных - задание выполняется в виде 5 отдельных программных проектов , где необходимо использовать:

\begin{tabular}{|l|c|c|c|c|c|}
   \hline
   \multicolumn{1}{|c|}{} & \multicolumn{5}{c|}{\textbf{Проект курсовой работыё}}\\
   \hline
   \textbf{Средства} & 1 & 2 & 3 & 4 & 5\\
   \hline
   массивы строк & - & + & - & - & -\\
   \hline
   массивы симоволов & - & + & - & - & -\\
   \hline
   внутренние процедуры головной программы & можно & + & + & - & -\\
   \hline
   Массив структур или структура массивов \textbf{выбрать} & - & - & + & + & -\\
   \hline
   файлы записей & - & - & + & + & -\\
   \hline
   модули & можно & можно & + & + & +\\
   \hline
   хвостовая рекурсия & - & - & - & + & +\\
   \hline
   однонаправленные списки заранее неизвестной длины & - & - & - & - & +\\
   \hline 
   регуляное программирование & + & + & + & + & +\\
   \hline
\end{tabular}


Дан список сотрудников научно-исследовательской лаборатории в виде:


\begin{tabular}{ll}
   ФАМИЛИЯ & ДОЛЖНОСТЬ\\ 15 симв. & 15 симв.\\
\end{tabular}

Пример входного файла:

\begin{tabular}{ll}
   Иванов & техник
\end{tabular}

Отсортировать список в порядке повышения должности от "техника" до "вед. инженера". Пример выходного файла:

\begin{tabular}{ll}
   Иванов & техник \\ Петров & старший инженер
\end{tabular}


Цель работы - выбор структуры данных для решения поставленной задачи на совеременных микроархитектурах. Задачи:
\begin{enumerate}
   \item Реализовать задание с использованием массивов строк.
   \item Реализовать задание с использованием массивов символов.
   \item Реализовать задание с использованием массива структур или структуры массивов.
   \item Реализовать задание с использованием хвостовой рекурсии
   \item Реализовать задание с использованием динамического списка
   \item Провести аназил на регулярный доступ к памяти
   \item Провести анализ на векторизацию кода
   \item Провести сравнительный анализ реализаций
\end{enumerate}
