
\title{Глава1}

\chaptername{ Реализация и анализ применения различных структур данных}
\section{Реализация массива строк}
 Данной реализации было создано несколько массивов строк для фамилий, специальнойстей и для карьерного списка и две строки для сортировки.
\begin{lstlisting}
   character(SURNAME_LEN, kind=CH_)                :: tmpSurname = "", Surnames(AMOUNT) = ""
   character(POST_LEN, kind=CH_)                   :: tmpPost = "", Post(AMOUNT) = "",Doljnost(4) = ""
\end{lstlisting}
   Сортировка производится методом пузырька. Поиск людей который необходимо поменять местами осуществляется следующим образом:
\begin{lstlisting}
         if (FindLoc(Doljnost,Post(j),dim=1)>Findloc(Doljnost,Post(j+1),dim =1)) then
            Swap = .true.
            else if (Surnames(j)>Surnames(j+1) .and. Post(j) == Post(j+1)) then
               Swap = .true.
         end if
\end{lstlisting}
  Здесь же осуществляется регулярный доступ к данным.

\section{Реализация массива символов}
В данной реализации необходимо обратить на следующий фрагмент, согласно которому считанные фамиилии и долности будут храниться в "столбцах":
\begin{lstlisting}
   character(kind=CH\_)                ::  Surnames(SURNAME_LEN,AMOUNT) = ""
   character(kind=CH\_)                ::  Post(Post_LEN,AMOUNT) = "",Doljnost(POST\_LEN,DOLJNOST\_LEN) = ""
\end{lstlisting}
Распологая данные таким образом мы предоставляем регулярный доступ к данным при сортривке.
Сортировка производится так же методом пузырька, проверка производится таким образом:
\begin{lstlisting}
         Swap = swap_doljnost(Doljnost,Post(:,j),Post(:,j+1))
         if(.not. Swap .and. GT(Surnames(:,j),Surnames(:,j+1)) .and. All(Post(:,j)==Post(:,j+1))) Swap = .true.
    
    pure logical function swap_doljnost(Doljnost,arr1,arr2)
      character (kind=CH_) arr1(:), Doljnost(:,:),arr2(:)
      intent (in) arr1,Doljnost,arr2
      integer i,j,k
      swap_doljnost = .false.
      j=0
      k=0
      do  i=1, DOLJNOST\_LEN
         if (ALL(arr1(:)==Doljnost(:,i)))j=i
         if (ALL(arr2(:)==Doljnost(:,i)))k=i
      end do
      swap_doljnost = j>k
   end function swap_doljnost

   pure logical function GT(arr1, arr2)
    character(kind=CH_), intent(in) :: arr1(:), arr2(:)

    integer :: i

    do i = 1, Min(Size(arr1), Size(arr2)) - 1
       if (arr1(i) /= arr2(i)) &
          exit
    end do
    GT = arr1(i) > arr2(i)
 end function GT
\end{lstlisting}
 Благодаря правильному расположению данных в массивах осуществляется регулярный доступ к памяти при перемещении фамилий
\begin{lstlisting}
         if (Swap) then
            tmpSurname        = Surnames(:,j+1)
            Surnames(:,j+1)   = Surnames(:,j)
            Surnames(:,j)     = tmpSurname

            tmpPost           = Post(:,j+1)
            Post(:,j+1)       = Post(:,j)
            Post(:,j)         = tmpPost
         end if
\end{lstlisting}
\section{Реализация массива структур}
\section{Реализация рекурсивных процедуры в массиве структур}
\section{динамического списка}
